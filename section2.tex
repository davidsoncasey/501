\section{Background in Hermitian Geometry}

In this section, we'll define and introduce introduce several objects that will be used throughout the paper.
Some definitions:

\begin{itemize}
\item A ring is said to be \textbf{local} if it has a unique maximal left ideal or unique maximal right ideal.
\item The \textbf{Jacobson radical} of a ring $R$, denoted $J(R)$ is the intersection of all maximal left (right) ideals.
In a local ring, $J(R)$ coincides with the unique maximal left ideal and unique maximal right ideal, showing that the maximal ideal is two sided.
\end{itemize}
Throughout the paper, the following objects will be fixed:
\begin{itemize}
\item Let $A$ be a local ring with identity.

\item Let $\mathfrak{r}$ be the Jacobson radical of $A$. Because $A$ is local, $\mathfrak{r}$ is maximal, two-sided, and contains all non-units of $A$.

\item Let $\Astar$ denote the set of all units of $A$.

\item Let $*$ be an involution of $A$. Assume that elements fixed by $*$ are in the center of $A$, forming a ring $R = \{a \in A: a^* = a\}$. 

\item Note that $R$ is local as well, with maximal ideal $\frakm = R \cap \mathfrak{r}$. 
This is because any element of $R$ that is not in $\frakm$ is invertible by definition, and cannot be contained in any proper ideal.

\item Let $\Rstar$ denote the set of all units of $R$.

\item Let $Q: A \to R: a \mapsto aa^*$ denote the norm-map.
\end{itemize}

\begin{example}\label{ex2.1}
Returning to the objects defined in the introduction, we can see an example of each of these objects using the \padic integers.
Let $\mathbb{Q}_p, \mathcal{O}, \mathcal{R}$ and $\omega$ be the same as defined in the previous section.
For simplicity, assume that $p > 2$ and $p \equiv 3 \mod 4$.
Then $\omega = \sqrt{p}$.
For $a + b \sqrt{p} \in \mathcal{R}$, let $\overline{a + b\sqrt{p}}$ denote the coset $a + b\sqrt{p} + (p)^{2\ell}$ in $A = \mathcal{R} / (\sqrt{p})^{2\ell}$.
Then
\begin{itemize}
\item $A = \mathcal{R}/(\sqrt{p})^{2\ell}$ is a local ring with maximal ideal $\frakr = (\sqrt{p})$.

\item $\Astar = A \backslash (\sqrt{p}) = \{\overline{a + b \sqrt{p}} : a \not\in (p) \}$.

\item $*: A \to A$ is defined by $\overline{a + b\sqrt{p}} \mapsto \overline{a - b\sqrt{p}}$.

\item $R = \{\overline{a + b\sqrt{p}} : b = 0\}$ with maximal ideal $\frakm = (p)$.

\item $\Rstar = \{\overline{a + b\sqrt{p}}: b = 0 \text{ and } a \not\in (p)\}$.

\item $Q: \Astar \to \Rstar: \overline{a + b\sqrt{p}} \mapsto \overline{a^2 - b^2 p}$.
\end{itemize}
The definition of these objects will remain constant through examples for the remainder of the paper.
\end{example}

Let $V$ be a right $A$-module. 
A function $h: V \times V \to A$ is a \textbf{Hermitian form} if
\begin{itemize}
\item $h$ is linear in the second variable.
That is, for $u, v, w \in V$ and $a \in A$, $h(u, v + w) = h(u, v) + h(u, w)$ and $h(u, va) = h(u, v)a$
\item $h(u,v) = h(v, u)^*$.
\end{itemize}
Combining these facts shows that $h(ua, v) = a^* h(u, v)$ for all $u, v \in V$ and $a \in A$.
Also, $h(u,u) = h(u,u)^*$ and $h(u,u) \in R \subseteq Z(A)$ for all $u \in V$.

Now consider the dual space $V^*$. Define an operation $V^* \times A \to V^*$ by $(\alpha a)(v) = a^* \alpha(v)$ where $\alpha \in V^*, a \in A, v \in V$.
Under this operation, $V^*$ is a right $A$-module.
Now we can define a homomorphism of right $A$-modules $\gamma_h: V \to V^*$ associated with $h$ given by $\gamma_h(u) = h(u, -)$.
Additionally, for the remainder of the paper:
\begin{itemize}
\item Assume that $h$ is non-degenerate; By definition, $\gamma_h$ is an isomorphism.
\item Let $U$ be the subgroup of $GL(V)$ preserving $h$. That is, for $\varphi \in U, u,v \in V$, $h(\varphi(u), \varphi(v)) = h(u,v)$.
\item Assume the existence of an element $d \in A$ such that $d + d^* = 1$.
\item Assume that $V$ is a free $A$-module of rank $m \ge 1$.
\end{itemize}
For the remainder of this section, let $\{v_1, v_2, \dotsc, v_m\}$ be a basis of $V$.

\begin{example}\label{ex2.2}
Continuing with the $A$ as defined in \cref{ex2.1} and assuming that $V$ is a right $A$-module,
let $a = v_1 a_1 + \dotsb v_m a_m$ and $b = v_1 b_1 + \dotsb + v_m b_m$ with $a_i, b_i \in A$ for $1 \le i \le m$.
Define $h: V \times V \to A$ by $h(a, b) = a_1^* b_1 + \dotsc + a_m^* b_m$.
This is a Hermitian form, and can be verified to be non-degenerate.
For the rest of the paper, assume that this is how $h$ is defined in all examples.
\end{example}

% Lemma 2.1
\begin{lemma}\label{lemma2.1}
There is a vector $u \in V$ such that $h(u,u) \in \Rstar$.
\end{lemma}
\begin{proof}
Assume otherwise; that  $h(u,u) \in \mathfrak{m}$ for all $u \in V$.
Then using the linearity of $h$:
\[
h(u,v) + h(u,v)^* = h(u+v, u+v) - h(u,u) - h(v,v) \in \mathfrak{m}
\]
for all $u,v \in V$.
Let $\alpha \in V^*$ be the linear functional such that $\alpha(v_1) = d$ and $\alpha(v_i) = 0$ for all $i > 1$.
Because $h$ is assumed to be non-degenerate, there exists $u \in V$ such that $h(u,-) = \alpha$.
Then $d = \alpha(v_1) = h(u,v_1)$ and $1 = d + d^* = h(u,v_1) + h(u,v_1)^* \not\in \mathfrak{m}$, contradicting the original hypothesis.
\end{proof}

\begin{example}
Note that in the $\mathbb{Q}_p$, $2^{-1}$ is a series satisfying $2^{-1} + 2^{-1} = 1$.
Because $2^{-1}$ is a \padic integer, $2^{-1} \in R$ and is fixed under the involution.
Thus $d = 2^{-1}$.
Using $h$ as defined in \cref{ex2.2}, pick $u = v_1 2^{-1}$.
Then $h(u, v_1) = 2^{-1}$ and $h(u, v_i) = 0$ for $i \ne 1$, as desired.
\end{example}

% Lemma 2.2
\begin{lemma}\label{lemma2.2}
$V$ has an orthogonal basis $u_1, u_2, \dotsc u_m$.
Any such basis satisfies $h(u_i, u_i) \in \Rstar$.
\end{lemma}
\begin{proof}
Prove with induction on $m$.
Assume that $m = 1$.
By \cref{lemma2.1}, there exists $u \in V$ such that $h(u,u) \in \Rstar$.
Then $u = v_1a_1$ for some $a_1 \in \Astar$, and $h(u,u) = h(v_1a_1, v_1a_1) = a_1^*h(v_1,v_1)a_1 \in \Rstar$ implying that $h(v_1, v_1) \in \Rstar$.
Now assume that $m > 1$ and that the hypothesis holds for $m - 1$.
Once again, there exists $u \in V$ such that $h(u,u) \in \Rstar$.
Then $u = v_1a_1 + \dotsb + v_m a_m$ with $a_i \in A$.
If all $a_i \in \mathfrak{r}$, then using the properties of $h$:
\[
h(u,u) = \sum_{i=1}^m \sum_{j=1}^m a_i^* h(v_i, v_j) a_j
\]
and each term of this sum is an element of $\frakr$.
Because $\frakr$ is an ideal, $h(u,u) \in \frakr$ implying that $h(u,u) \in \frakr \cap R = \frakm$, a contradiction.
Without loss of generality, assume that $a_1 \not\in \mathfrak{r}$.
Then if $u_1 = v_1 a_1$, the set $\{u_1, v_2, \dotsc, v_m\}$ is a basis of $V$.
For $1 < i \le m$, set
\[
u_i = v_i - u_1[h(u_1,v_i)/h(u_1, u_1)]
\]
Then $u_1, u_2, \dotsc, u_m$ can be shown to be a basis of $V$ satisfying $h(u_1, u_i) = 0$ for $1 < i \le m$.
Let $V_1 = u_1 A$ and $V_2 = \text{span} \{u_2, \dotsc, u_m\}$.
Then $V = V_1 \perp V_2$ and the restriction of $h$ to $V_2$ induces an isomorphism $V_2 \to V_2^*$.
Applying the inductive hypothesis to this space completes the proof.
\end{proof}

% Lemma 2.3
\begin{lemma}\label{lemma2.3}
\begin{description}
% Come back to this to clean up the latex
\item{(a)} Suppose $u_1, \dotsc, u_s \in V$ are orthogonal and satisfy $h(u_i, u_i) \in \Rstar$.
Then $u_1, \dotsc , u_s \in V$ can be extended to an orthogonal basis of $V$ with the same property.
\item{(b)} If $V_1$ is a submodule of $V$ such that the restriction of $h$ to $V_1$ is non-degenerate there is another such submodule $V_2$ of $V$ such that $V = V_1 \perp V_2$.
\end{description}
\end{lemma}
\begin{proof}
(a) Because $\{v_1, \dotsc, v_m \}$ is a basis of $v$, $u_1 = v_1 a_1 + \dotsb + v_m a_m$ for some $a_i \in A$.
Since $h(u_1, u_1) \in \Rstar$ (by hypothesis), one of the scalars must be a unit (see proof of \cref{lemma2.2}).
Without loss of generality, assume $a_1 \in \Astar$.
Thus $u_1, v_2, \dotsc, v_m$ is a basis of $V$.
Suppose $1 \le t \le s$ and the list $u_1, \dotsc, u_t, v_{t+1}, \dotsc, v_m$ is a basis of $V$.
Then 
\[
u_{t+1} = u_1b_1 + \dotsb + u_t b_t + v_{t+1} b_{t+1} + \dotsb + v_m b_m
\]
for some $b_i \in A$.
Suppose, if possible, that $b_i \in \mathfrak{r}$ for all $i \ge t+1$.
Then for every $i \le t$,
\[
0 = h(u_i, u_{t+1}) = h(u_i, u_i) b_i + h(u_i, v_{t+1})b_{t+1} + \dotsb + h(u_i, v_m) b_m
\]
implying that $b_i \in \mathfrak{r}$ for all $1 \le i \le t$, contradicting the assumption that $h(u_{t+1}, u_{t+1}) \in \Rstar$.
Thus at least one of $b_{t+1}, \dotsc,b_m$ is a unit (assume $b_{t+1}$ and $u_1, \dotsc u_t, u_{t+1}, v_{t+2}, \dotsc, v_m$ is a basis of $V$.

This process can be repeated to extend $u_1, \dotsc, u_s$ to a basis $u_1, \dotsc, u_s, u_{s+1}, \dotsc u_m$ of $V$.
For $s < i \le m$, let
\[
z_i = u_i - ([u_1h(u_1, u_i)/h(u_1, u_1)] + \dotsb + u_s h(u_s, u_i) / h(u_s, u_s)].
\]
Then $u_1, \dotsc, u_s, z_1, \dotsc, z_{m-s}$ is a basis of $V$ satisfying $h(u_i, z_j) = 0$.
If follows that the restriction of $h$ to $M = \text{span} \{z_1, \dotsc, z_{m-s}\}$ is non-degenerate and by \cref{lemma2.2} that $M$ has an orthogonal basis with $h(z_i, z_i) \in \Rstar$ for any $i \le m - s$.

(b) Follows from (a) and \cref{lemma2.2}
\end{proof}

Given a set $U = \{u_1, \dotsc, u_s\} \subset V$, for $s \le m$, the \textbf{Gram matrix} $H$ of the set $U$ with respect to $h$
is defined by $H_{ij} = h(u_i, u_j)$.
If $U$ is a basis, then for any $u, v \in V$, $h(u,v) = u^*Hv$, where $u^*$ is the transpose of $u$ under $*$.

% Lemma 2.4
\begin{lemma}\label{lemma2.4}
Let $u_1, \dotsc u_s \in V$, with corresponding Gram matrix $H \in M_s(A)$, defined by $H_{ij} = h(u_i, u_j)$.
If $H \in GL_m(A)$, then $u_1, \dotsc, u_s$ are linearly independent.
\end{lemma}
\begin{proof}
Suppose $a_1, \dotsc, a_s$ satisfy $u_1 a_1 + \dotsb + u_sa_s = 0$.
Then for $1 \le i \le s$
\[
0 = h(u_i, u_1 a_1 + \dotsb + u_s a_s) = h(u_i, u_1)a_1 + \dotsb + h(u_i, u_s)a_s
\]
implying that 
\[
H \left( \begin{array}{c}
a_1 \\
\vdots \\
c_1
\end{array} \right)
= \left( \begin{array}{c}
0 \\
\vdots \\
0
\end{array}
\right).
\]
Since $H$ is invertible, the desired result follows.
\end{proof}
