\documentclass[11pt]{article}
\usepackage{amsmath}
\usepackage{amssymb}
\usepackage{natbib}
\newcommand{\padic}{$p$-adic }
\begin{document}
\section{Introduction Workthrough with \padic Numbers}
$K$ is a non-archimedian local field. In this case $K$ will be the \padic numbers $\mathbb{Q}_p$, where $p > 1$ is prime.
Let 	$\mathcal{O}$ be the ring of integers (integral elements) of $K$. These are elements of the form
\[
a = \sum_{i > 0}a_i p^i.
\]
with each $0 \le a_i \le p - 1$ for all $i$. See \cite{katok}, page 28 for more explanation of this representation.
It is important to note that because of the metric imposed on the field of \padic numbers, adding terms with higher powers of $p$ does not increase the magnitude of the integer.
Thus for every \padic integer $a$, $|a| \le 1$.
As in the rational numbers (can be shown with rational root theorem) these are the integers.
To see that $(p)$ is the maximal ideal, note that for any $a \in \mathbb{O}$, if $|a| = 1$, then $a$ is invertible.
Thus the only non-invertible elements of $\mathbb{O}$ have $p$ as a divisor, and $(p)$ is the unique maximal ideal of $\mathbb{O}$.

Now let $F_q = \mathbb{O}/(p)$. Then this is a field of characteristic $p$.
Let $F = K[\sqrt{p}]$ be a ramified quadratic extension of $K$ with ring of integers $R$.
From wikipedia \cite{wikipedia:quadratic-integer}: Define
\[
\omega = \begin{cases}
\sqrt{p} &\text{ if } p \equiv 2,3 \pmod{4}\\
\frac{1 + \sqrt{p}}{2} &\text{ if } p \equiv 1 \pmod{4}
\end{cases}
\]
Then $R = \{a + b \omega \} = \mathbb{O}[\sqrt{p}]$.

\bibliographystyle{plain}
\bibliography{501}
\end{document}