\begin{center}
\section{An Introduction to Local Rings}
\end{center}

For my 501 project, I'll be working through the paper ``Unitary Groups Over Local Rings"\cite{cruickshank}.
The paper contains a discussion of hermitian forms and unitary groups over a local ring.
In my project, I'll provide an introduction to the properties of local rings, and a detailed discussion of hermitian forms
defined on module of a local ring.

This introduction gives a brief discussion of local fields and the construction of a local ring $A$ from a local field $K$.
In this example $K$ will be taken to be the \padic numbers, but this construction generalizes to any local field.
First, some definitions:

\begin{itemize}[noitemsep]
\item A \textbf{local field} is a locally compact topological field with respect to a non-discrete topology.
This allows the field to be equipped with an absolute value.
The real numbers $\mathbb{R}$ and rational numbers $\mathbb{Q}$ are examples of local fields.
\item A local field $K$ is said to be \textbf{archimidean} if for any element $x \in K$, there exists $n$ such that $|nx| > 1$.
A local field is said to be \textbf{non-archimidean} if this property does not hold.
\item In any field, the \textbf{ring of integers} refers to the set of all integral elements, that is, 
elements that are the root of a monic polynomial with integer coefficients.
\item Given a ring $A$, an automorphism $*: A \to A$ denoted ($a \mapsto a^*$ for $a \in A$) is an \textbf{involution} 
if it satisfies $(a^*)^* = a$ for all $a \in A$ and $(ab)^* = b^* a^*$ for $a, b \in A$.
\end{itemize}

Given a prime number $p$, consider the representations of the integers $\mathbb{Z}$ base $p$.
That is, for any integers $a$, $a$ can be expressed as a sum
\[
a = \sum_{i=0}^n a_i p^i
\]
where $0 \le a_i \le p - 1$.
The \padic norm, $|\cdot|_p$, can be applied to this representation, defined by
\[
|x|_p = \begin{cases}
p^{-n_p(x)} &\text{ if } x \ne 0\\
0 &\text{ if } x = 0
\end{cases}
\]
where $n_p(x) = \max{n: p^n | x}$ if $p | x$ and $n_p(x) = 1$ if $p \nmid x$.
This can be shown to satisfy the definition of a norm.
Using this norm, the set completion of all expansions of the form
\[
a = \sum_{i=k}^\infty a_i p_i
\]
(with $k$ not necessarily positive) can be shown to be a field, known as the \padic numbers (denoted $K$) for the remainder of the section.
The integers form a ring of integers, $\mathcal{O}$, 
with maximal ideal $\mathfrak{p} = (p)$ and residue field $F_q = \mathcal{O}/\mathfrak{p}$ of characteristic $p$.
Let $F = K[\sqrt{p}]$.
Depending on $p \mod 4$, the ring of integers is slightly different (see \cite{milneANT} or \cite{samuel} for a more detailed discussion).
Define
\[
\omega = \begin{cases}
\sqrt{p} &\text{ if } p \equiv 2,3 \pmod{4}\\
\frac{1 + \sqrt{p}}{2} &\text{ if } p \equiv 1 \pmod{4}
\end{cases}
\]
Then the ring of integers $\mathcal{R} = \mathcal{O}[\omega]$.
Then $\mathcal{R}$ is a free $\mathcal{O}$-module of rank $2$.
$\mathcal{R}$ is a principle ideal domain with a unique prime element $\omega$ (up to units), and $\mathcal{R}/(\omega) \cong F_q$.
There is an involution of $F$ defined by $a + b \omega \mapsto a - b \omega$ that fixes $K$, with $\mathcal{R}$ invariant under this involution.
Given $\ell \ge 1$, let $A = \mathcal{R} / (\omega)^{2\ell}$.
Then $A$ is a finite, commutative, principal ring, with an involution inherited from $\mathcal{R}$.
$A$ is also a local ring, a concept that will be introduced in detail in the next section.
