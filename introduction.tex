\section{Introduction}

For my 501 project, I'll be working with the paper ``Unitary Groups Over Local Rings"\cite{cruickshank}.
The paper contains a discussion of hermitian forms and unitary groups over a local ring.
In my project, I'll provide an introduction to the properties of local rings, and a detailed discussion of hermitian forms
defined on modules over a local ring.
Several results are presented that investigate properties of such modules, as well as specific properties of hermitian forms on these modules,
and relationships between norm maps and hermitian geometry.

\subsection{Basics of Local Rings}

This introduction gives a brief discussion of local fields and the construction of a local ring from a local field.
First, some definitions:

\begin{itemize}
\item A field $K$ is a \textbf{local field} if it is complete with respect to a discrete valuation and a has finite residue field.
An equivalent definition is that a field is locally compact with respect to a non-discrete topology (see \cite{serre} for a detailed explanation).
This allows an absolute value function $|\cdot\|$ to be defined.
The real numbers $\mathbb{R}$ and rational numbers $\mathbb{Q}$ are examples of local fields, with the standard topology and absolute values.

\item A local field $K$ is said to be \textbf{archimidean} if for any element $x \in K$, there exists $n \in \mathbb{N}$ such that $|nx| > 1$.
A local field is said to be \textbf{non-archimidean} if this property does not hold.

\item In any field, the \textbf{ring of integers} refers to the set of all integral elements, that is, 
elements that are the root of a monic polynomial with integer coefficients 
(that is, coefficients from the set $\{z \cdot 1_K : z \in \mathbb{Z} \}$, where $1_k$ is the multiplicative identity of $K$).

\item Given a ring $A$, an automorphism $*: A \to A$ (denoted $a \mapsto a^*$ for $a \in A$) is an \textbf{involution} 
if it satisfies $(a^*)^* = a$ for all $a \in A$ and $(ab)^* = b^* a^*$ for $a, b \in A$.
\end{itemize}

Given a prime number $p$, consider the representations of the integers $\mathbb{Z}$ base $p$.
That is, for any integer $a$, $a$ can be expressed as a sum
\[
a = \sum_{i=0}^n a_i p^i
\]
where $0 \le a_i \le p - 1$.
The \padic norm, $|\cdot|_p$, can be applied to this representation, defined by
\[
|x|_p = \begin{cases}
p^{-n_p(x)} &\text{ if } x \ne 0\\
0 &\text{ if } x = 0
\end{cases}
\]
where $n_p(x) = \max \{n: p^n | x \}$ if $p | x$ and $n_p(x) = 1$ if $p \nmid x$.
This can be shown to satisfy the definition of a norm.
Using this norm, every element $a$ of the form
\[
a = \sum_{i=k}^\infty a_i p_i \;\;\; (k \text{ not necessarily positive})
\]
can be shown to be a field, known as the \padic numbers (denoted $\mathbb{Q}_p$ for the remainder of the section).
The standard integers $\mathbb{Z} \subset \mathbb{Q}_p$ form a ring of integers, denoted $\mathcal{O}$, 
with maximal ideal $(p)$ and residue field $\mathbb{F}_p = \mathcal{O}/(p)$ of characteristic $p$.
Unlike the standard integers, elements of $\mathcal{O} \backslash (p)$ are invertible (see \cite{katok} for an explanation).

Define $F = \mathbb{Q}_p [\sqrt{p}]$.
Depending on $p \mod 4$, the ring of integers of $F$ is slightly different (see \cite{milneANT} or \cite{samuel} for a more detailed discussion).
Define
\[
\omega = \begin{cases}
\sqrt{p} &\text{ if } p \equiv 2,3 \pmod{4}\\
\frac{1 + \sqrt{p}}{2} &\text{ if } p \equiv 1 \pmod{4};
\end{cases}
\]
then the ring of integers of $\mathbb{Q}_p$ is $\mathcal{R} = \mathcal{O}[\omega]$.
Then $\mathcal{R}$ is a free $\mathcal{O}$-module of rank $2$.
$\mathcal{R}$ is a principle ideal domain with a unique prime element $\omega$ (up to associates), and $\mathcal{R}/(\omega) \cong \mathbb{F}_p$.
There is an involution of $F$ defined by $a + b \omega \mapsto a - b \omega$ that fixes $\mathbb{Q}_p$, with $\mathcal{R}$ invariant under this involution.
Given $\ell \ge 1$, let $A = \mathcal{R} / (\omega)^{2\ell}$.
Then $A$ is a finite, commutative, principal ideal ring, with an involution inherited from $\mathcal{R}$.
$A$ is also a local ring, a concept that will be introduced in detail in the next section.


