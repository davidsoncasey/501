\documentclass[11pt]{article}
\usepackage{project501}
\begin{document}
\begin{titlepage}
\begin{center}

\textsc{\LARGE Portland State University}\\[1.5cm]

\textsc{\Large Math 501 Project}\\[0.5cm]

% Title
\hrule ~\\[0.4cm]
{ \large \bfseries Properties of Hermitian Forms on Modules over Local Rings \\ }
{ \bfseries A short presentation \\[0.4cm] }

\hrule ~\\[0.75cm]

\begin{center}
\emph{Author:}\\
Casey \textsc{Davidson}\\[1cm]

\emph{Supervisor:}\\
Dr.~Joyce \textsc{O'Halloran}\\[1cm]

\emph{Second Reader:}\\
Dr.~Derek \textsc{Garton}
\end{center}

\vfill

% Bottom of the page
{\large \today}

\end{center}
\end{titlepage}

\section*{Local Fields}
\subsection*{An Introduction to Algebraic Structures}

% Definition of a group
\begin{definition*}
A \textbf{group} is a nonempty set $G$ equipped with a binary operation $+$ on $G$ such that:
\begin{enumerate}
\item $+$ is associative: $(a + b) + c = a + (b + c)$ for all $a, b, c \in G$.
\item There exists an element $0_G$ such that $a + 0_G = 0_G + a = a$ for all $a \in G$. 
This is known as the \textbf{additive identity}.
\item For every $a \in G$, there exists an element $-a$ such that $-a + a = a + (-a) = 0_G$.
This is known as the \textbf{inverse element}.
\end{enumerate}
If, additionally, $a + b = b + a$ for all $a, b \in G$, then the $G$ is an \textbf{abelian group}.
\end{definition*}

% Definition of a ring
\begin{definition*}
A \textbf{ring} is a nonempty set $R$, equipped with two binary operations, $+$ and $\cdot$ such that:
\begin{enumerate}
\item $(R, +)$ is an abelian group, with identity element $0_R$.
\item $a \cdot b$ (denoted $ab$) is associative. That is, for all $a, b, c \in R$, $(ab)c = a(bc)$.
\item $a(b + c) = ab + ac$ and $(a + b)c = ac + bc$ (distributive property of multiplication).
\end{enumerate}
If $ab = ba$ for all $a, b \in R$, then $R$ is said to be \textbf{commutative}.
If there exist an element $1_R$ such that $a 1_R = 1_R a = a$ for all $a$, then $1_R$ is the \textbf{multiplicative identity}
and $R$ is said to be a \textbf{ring with identity}
\end{definition*}
\begin{example*}
The integers $\mathbb{Z} = \{\dotsc -2, -1, 0, 1, 2, \dotsc \}$ are a commutative ring with identity, with the standard operations of addition and multiplication.
\end{example*}

% Definition of an involution
\begin{definition*}
Let $R$ and $S$ be rings.
A mapping $\phi: R \to S$ is said to be a \textbf{ring homomorphism} provided that:
\begin{enumerate}
\item $\phi(a + b) = \phi(a) + \phi(b)$
\item $\phi(ab) = \phi(a) \phi(b)$
\end{enumerate}
for all $a, b \in R$.
An anti-homomorphism $\sigma: R \to R$ is a mapping such that addition is preserved as with a homomorphism, but $\sigma(ab) = \sigma(b) \sigma(a)$.
A bijective homomorphism is an \textbf{isomorphism}, and an isomorphism from $R$ onto itself is an \textbf{automorphism}.
An \textbf{anti-automorphism} is defined analogously.
An anti-automorphism $*: R \to R: r \mapsto r^*$ for $r \in R$ is an \textbf{involution} if $(r^*)^* = r$.
\end{definition*}

% Definition of a field
\begin{definition*}
If $a$ is an element of a ring $F$ such that there exists an element $a^{-1}$ where $a a^{-1} = a^{-1}a = 1_F$, then $A$ is said to be invertible.
If every element of $F$ is invertible, and $1_F \ne 0_F$, then $F$ is called a \textbf{division ring}.
A \textbf{field} is a commutative division ring.
\end{definition*}
\begin{example*}
The rational numbers $\mathbb{Q}$ and the real numbers $\mathbb{R}$ are fields.
\end{example*}

\subsection*{Local Fields}
A local field is a special case of a field where the algebraic properties allow it to be equipped with an absolute value function, and the field is complete and locally compact with respect to this function.
\begin{definition*}
Let $A$ be a ring, and $K$ be the field of fractions of $A$.
A function $\nu: K \to \mathbb{Z}$ is a \textbf{discrete valuation} provided that for $x, y \in K$:
\begin{enumerate}
\item $\nu(xy) = \nu(x) + \nu(y)$
\item $\nu(x + y) \ge \inf\{\nu(x), \nu(y)\}$
\end{enumerate}
By convention, $\nu(0) = \infty$.
Now let $0 \le a \le 1$ and define a function $\|\cdot \|: K \to \mathbb{R}$ by
\[
\|x\| = \begin{cases}
a^{\nu(x)} &\text{ if } x \ne 0\\
0 &\text{ if } x = 0.
\end{cases}
\]
Then $\| \cdot \|$ is an \textbf{absolute value} on $K$.

If $p \in A$ is a unique irreducible element (up to multiplication by a unit), then each element $a \in K$ 
can be written uniquely in the form $a = p^n b$, where $b$ is invertible.
Define $\nu_p: K \to \mathbb{R}$ by $\nu_p(a) = n$.
Then $\nu_p$ is a discrete valuation.
Taking $a = \frac{1}{p}$, $\| \cdot \|$ is the absolute value defined with respect to $\nu_p$.
Then if $K$ is complete with respect to this absolute value, and the residue field $A/(p)$ is finite, $K$ is locally compact.
In this case, we say that $K$ is a \textbf{local field}. 
See \citet{serre} for a more in depth exploration of the relationship between discrete valuations and local fields.
The absolute value is called \textbf{archimedean} if for any $x \in K$, there exists an integer $n \in \mathbb{N}$ such that 
$\|nx\| > 1$.
If this property is not met, then $\|\cdot \|$ is \textbf{non-archimedean}.
\end{definition*}
\newpage

% P-adic numbers
\subsection*{The P-adic Numbers}
\begin{example*}
Let $p \ge 2$ be a prime.
Consider the representations of the integers $\mathbb{Z}$ base $p$.
That is, for any integer $a$, $a$ can be uniquely expressed as a sum
\[
a = \sum_{i=0}^n a_i p^i
\]
where $0 \le a_i \le p - 1$.
The notation $a = \dotsc a_3 a_2 a_1 a_0$ is equivalent.


Let $\nu_p$ be the discrete valuation with respect to $p$,
and $\|\cdot \|_p$ be the corresponding absolute value, taking $a$ as $\frac{1}{p}$.
Then for any integer $a$, $\|a\|_p = \|a_0\|_p$.
If $\|a_0\| = 1$, then $a$ is invertible, and $a^{-1}$ is an integer as well, an infinite sequence to to left.
The integers with this representation are the \textbf{p-adic} integers, and are a ring with unique prime element $p$, up to multiplication by a units.

Now expand these series to the form
\[
a = \sum_{i = -k}^\infty a_i p^i
\]
where $k > - \infty$ and $0 \le a_i \le p - 1$.
These elements form a field, known as the \padic numbers and denoted $\mathbb{Q}_p$.
Because $Z/(p)$ is finite and $\mathbb{Q}_p$ is complete with respect to the absolute value $\|\cdot \|_p$, $\mathbb{Q}_p$ is a local field.

\end{example*}
\newpage

\section*{Local Rings}
\begin{definition*}
Let $R$ be a ring.
A subset $J \subset R$ is a left ideal if:
\begin{enumerate}
\item $a + b \in J$ for all $a, b \in J$
\item $ra \in J$ for $a, \in J$ and $r \in R$.
\end{enumerate}
An analogous definition holds for a right ideal.
If $J$ is both a left and right ideal, then $J$ is a two-sided ideal.
An ideal $J$ is nontrivial if $J \ne \{0\}$, and is proper if $J \ne R$.
A nontrivial ideal is \textbf{maximal} if it is not properly contained in a any proper ideal.
\end{definition*} 

\begin{example*}
In the \padic integers as defined previously, $(p)$ is a maximal ideal.
\end{example*}

\begin{definition*}
A ring $R$ is said to be \textbf{local} if it has a unique maximal left or unique maximal right ideal.
The \textbf{Jacobson radical} of $R$, denoted $J(R)$ is the intersection of all maximal left (right) ideals.
In a local ring, $J(R)$ coincides with the unique maximal left ideal and unique maximal right ideal, showing that the maximal ideal is two sided.
In a local ring, $J(R)$ contains all nonunits, so $A/J(R)$ is a division ring (meaning that every element of $A/R$ is invertible).
\end{definition*}

\subsection*{Constructing a local ring from the \padic numbers}
\begin{example*}
For simplicity, assume that $p \equiv 3 \mod{4}$.
While a similar construction can be done for $p \equiv 1 \mod{4}$, it is slightly simpler when $p \equiv 4 \mod{4}$.
The standard integers $\Z \subset \Qp$ form a ring of integers, with maximal ideal $(p)$.
Define $F = \Qp[\sqrt{p}]$.
Then $F$ has a ring of integers $\mathcal{R} = \Z[\sqrt{p}]$.
Then $\mathcal{R}$ has elements of the form $a + b\sqrt{p}$ for $a, b \in \Z$, and an involution $a + b \sqrt{p} \mapsto a - b \sqrt{p}$.
$\mathcal{R}$ is a principle ideal domain with a unique prime element $\sqrt{p}$ (up to associates),
and $\mathcal{R} / (\sqrt{p}) \cong \Z / (p) = \Z_p$
For $\ell \ge 1$, define $A_p = \mathcal{R} / (\sqrt{p})^{2\ell}$.
Then $A_p$ is a finite, commutative, principle ideal ring.
Additionally, all ideals of $a$ are of the form $(p)^k$, $1 \le k \le 2 \ell$, and $(\sqrt{p})$ is a unique maximal ideal.
Thus $A_p$ is a local ring, that inherits an involution for $\mathcal{R}$.
This construction can be used on any non-archimedian local field, using the irreducible element of the local field from which the valuation is defined.
\end{example*}
\newpage

\section*{Modules and Hermitian Forms}
\begin{definition*}
Let $R$ be a ring.
A set $V$ is a right $R$-module if $V$ is an abelian group and there exists a function $V \times R \to V$
(denoted $ur$) such that for $r, s \in R$ and $u, v \in V$:
\begin{enumerate}
\item $(u + v)r = ur + vr$
\item $u(r + s) = ur + us$
\item $(us)r = u(rs)$
\end{enumerate}
A right module is defined analogously, with multiplication taking the form $ur$.
A set subset $U$ of a right $R$-module $V$ is \textbf{linearly independent} if for $u_1, \dotsc, u_n \in U$ and $r_i \in R$,
$u_1 r_1 + \dotsc u_n r_n = 0$ implies that each $r_i$ is zero.
If $V$ is generated by $U$, then $U$ spans $V$.
A \textbf{basis} is a linearly independent spanning set.
Consequently, if $R$ has identity and $v 1_R = v$ for all $v \in V$, then each $v$ has a unique representation
\[
v = u_1 x_1 + \dotsb + u_n x_n
\]
for $x_i \in R$.
\end{definition*}

\begin{example*}
The standard coordinate system $\mathbb{R}^2$ is a module over the real numbers, with basis $(1, 0)$ and $(0, 1)$.
\end{example*}

\begin{definition*}
Let $A$ be a ring equipped with an involution $*$.
Let $V$ be a right $A$-module. 
A function $h: V \times V \to R$ is a \textbf{Hermitian form} if:
\begin{enumerate}
\item $h$ is linear in the second variable.
That is, for $u, v, w \in V$ and $a \in A$, $h(u, v + w) = h(u, v) + h(u, w)$ and $h(u, va) = h(u, v)a$
\item $h(u,v) = h(v, u)^*$.
\end{enumerate}
Combining these facts shows that $h(ua, v) = a^* h(u, v)$ for all $u, v \in V$ and $a \in A$.
Also, $h(u,u) = h(u,u)^*$ and $h(u,u)$ is fixed under $*$.

Now consider the dual space $V^*$. Define an operation $V^* \times A \to V^*$ by $(\alpha a)(v) = a^* \alpha(v)$ where $\alpha \in V^*, a \in A, v \in V$.
Under this operation, $V^*$ is a right $A$-module.
Now we can define a homomorphism of right $A$-modules $\gamma_h: V \to V^*$ associated with $h$ given by $\gamma_h(u) = h(u, -)$.
$h$ is \textbf{nondegenerate} if $\gamma_h$ is an isomorphism.
\end{definition*}
\begin{example*}
Continue with $A_p$ as defined previously and assume that $V$ is a right $A_p$-module with a basis $\{v_1, \dotsc, v_m\}$.
let $a = v_1 a_1 + \dotsb v_m a_m$ and $b = v_1 b_1 + \dotsb + v_m b_m$ with $a_i, b_i \in A$ for $1 \le i \le m$.
Define $h: V \times V \to A$ by $h(a, b) = a_1^* b_1 + \dotsc + a_m^* b_m$.
This is a Hermitian form, and can be verified to be non-degenerate.
\end{example*}

\newpage

\section*{Classification of Hermitian Forms}
\begin{definition*}
For this section, the following objects will be fixed:
\begin{itemize}
\item Let $A$ be a local ring with identity.

\item Let $\mathfrak{r}$ be the Jacobson radical of $A$. Because $A$ is local, $\mathfrak{r}$ is maximal, two-sided, and contains all non-units of $A$.

\item Let $\Astar$ denote the set of all units of $A$.

\item Let $*$ be an involution of $A$. Assume that elements fixed by $*$ are in the center of $A$, forming a ring $R = \{a \in A: a^* = a\}$. 

\item Note that $R$ is local as well, with maximal ideal $\frakm = R \cap \mathfrak{r}$. 
This is because any element of $R$ that is not in $\frakm$ is invertible by definition, and cannot be contained in any proper ideal.

\item Let $\Rstar$ denote the set of all units of $R$.

\item Let $Q: A \to R: a \mapsto aa^*$ denote the norm-map.

\item Let $V$ be a right $A$-module of dimension $m$.

\item Let $h: V \times V \to A$ be a Hermitian form.
\end{itemize}

A vector $v \in V$ is said to be \textbf{primitive} if $v \not\in V \frakr = \{wr : w \in V, r \in \frakr \}$.
This is equivalent to $v$ being an element of some basis.
If there exists a primitive $v$ such that $h(v,v) = 0$, then $h$ is \textbf{isotropic}.
\end{definition*}

% Lemma 3.1
\begin{lemma*}
Suppose $h$ is isotropic. Then, given any $r \in R$ there is a primitive vector $v$ satisfying $h(v,v) = r$.
\end{lemma*}

\begin{proposition*}
The division ring $\Amodr$ is commutative. Moreover,
\begin{description}
\item[(a)] If the involution that $*$ induces on $\Amodr$ is the identity then $Q$ is not surjective and $\Amodr \cong F_q$.
\item[(b)] If the involution that $*$ induces on $\Amodr$ is not the identity then $Q$ is surjective and $\Amodr \cong F_{q^2}$.
\end{description}
\end{proposition*}

\begin{example*}
Recall that in our example, $\frakr = (\sqrt{p})$ and $\frakm = (p)$.
Thus $A/\frakr \cong R/\frakm \cong F_p$.
Furthermore, the involution $*$ induces on $\Amodr$ is the identity.
Pick $c \in F_p \backslash F_p^2$.
Then $c$ cannot be in the image of $*$, because for any $a + b\sqrt{p} \in A$,
$c \not\equiv a^2 \mod p$, and therefore $c \ne a^2 - b^2 p$.
Thus $Q$ is not surjective.
\end{example*}

% Theorem 3.5
\begin{theorem*}
There is an orthogonal basis $v_1, v_2, \dotsc, v_m$ of $V$ satisfying
\begin{align*}
  h(v_1, v_1) &= \dotsb = h(v_{m-1}, v_{m-1}) = 1 \text{ and } \\
  h(v_m, v_m) &= 1 \text{ if } Q(\Astar) = \Rstar \\
  h(v_m, v_m) &\in \{1, \varepsilon \} \text{ if } Q(\Astar) = {\Rstar}^2
\end{align*}
\end{theorem*}

\begin{definition*}
Given $r_1, \dotsc, r_m \in \Rstar$ we say that $h$ is of type $\{r_1, \dotsc, r_m \}$ if there is a basis $B$ of $V$ relative to which $h$ has matrix $\text{diag} \{r_1, \dotsc, r_m \}$.
Note that because these matrices contain only elements of $\Rstar$, which is commutative, the notion of a determinant is well defined.
\end{definition*}

\begin{lemma*}
When $m$ is even then $h$ is of type $\{1, -1, \dotsc, 1, -1 \}$ (define this as kind I) or $\{1, -1,\dotsc, 1, -\varep \}$ (kind II).
When $m$ is odd then $h$ is of type $\{1, -1, \dotsc, 1, -1, -1 \}$ (kind I) or of type $\{1, -1, \dotsc, 1, -1, -\varep \}$ (kind II).
\end{lemma*}






\newpage
\nocite{hungerford}
\nocite{katok}
\bibliographystyle{plainnat}
\bibliography{501}




\end{document}
